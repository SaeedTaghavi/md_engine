%
% Copyright (c) 2013, Daniel Guterding <guterding@itp.uni-frankfurt.de>
%
% This file is part of vsmd.
%
% vsmd is free software: you can redistribute it and/or modify
% it under the terms of the GNU General Public License as published by
% the Free Software Foundation, either version 3 of the License, or
% (at your option) any later version.
%
% vsmd is distributed in the hope that it will be useful,
% but WITHOUT ANY WARRANTY; without even the implied warranty of
% MERCHANTABILITY or FITNESS FOR A PARTICULAR PURPOSE. See the
% GNU General Public License for more details.
%
% You should have received a copy of the GNU General Public License
% along with vsmd. If not, see <http://www.gnu.org/licenses/>.
%

\documentclass[a4paper,11pt]{scrartcl}
\usepackage[english]{babel}
\usepackage{amssymb,amsmath}
\usepackage[utf8x]{inputenc}

\begin{document}
\title{Lennard-Jones Reduced Units}
\date{}
\maketitle

\section*{Why use reduced units ?}
Reduced units divide all physical quantities by characteristic scales of the simulated system. Thus all simulated quantities will take values around unity in reduced units. This avoids numerical errors due to very small or large numbers and makes coding errors that result in unphysical behaviour easier to spot.

\section*{Example}
As an example we take the Lennard-Jones 12-6-Potential and transform all quantities to their starred reduced analogs. The characteristic scale of the energy is the depth of the potential $\varepsilon$. The characteristic scale of the length is the root $\sigma$. The Potential $\phi$ is an energy, so $\phi^* = \phi / \varepsilon $. The pair distance $r$ is a length, so $r^* = r / \sigma $
\begin{displaymath}
\phi (r) = 4 \varepsilon \left[ \left( \frac{\sigma}{r} \right)^{12} - \left( \frac{\sigma}{r} \right)^6 \right] \quad \rightarrow \quad \phi^* (r^*) = 4 \left[ (r^*)^{-12} - (r^*)^{-6} \right]
\end{displaymath}

\begin{table}[h!]
\centering
\begin{tabular}{lrr}
quantity & symbol & definition \\
\hline
length & $r^*$ & $r / \sigma$  \\
energy & $E^*$ & $E / \varepsilon$ \\
mass & $m^*$ & 1\\
time & $t^*$ & $t / \left( \sigma \sqrt{m / \varepsilon} \right)$ \\
temperature & $T^*$ & $k_B T / \varepsilon$ \\
number density & $\rho^*$ & $\rho \sigma^3$ \\
velocity & $v^*$ & $v / \sqrt{\varepsilon / m}$  \\
force & $F^*$ & $F \sigma / \varepsilon$ \\
pressure & $p^*$ & $p \sigma^3 / \varepsilon$ \\
\end{tabular}
\caption{Conversion table for Lennard-Jones reduced units. Starred letters are associated with reduced units.}
\end{table}

\section*{Parameter sets}
The results obtained in reduced units can be interpreted in a physically meaningful way through the use of parameters $\varepsilon$ and $\sigma$ that reflect properties of real systems. For the most well know set, the so called \textit{Bernardes parameters}, see  Bernardes, N.: Phys. Rev. 112 (1958), p. 1534–1539.

\end{document}